\newpage
\section{Theorie}
\subsection{Grundlagen}
Kriterien für Einteilung Energieversorgungsnetze:
\begin{itemize}
    \item Spannungsart \item Spannungshöhe \item Funktion \item Topologie
\end{itemize}
% - Spannungsart \qquad - Spannungshöhe\\
% - Funktion \qquad \qquad- Topologie

Arten der Netztopologien:
\begin{itemize}
\item Strahlennetz (z.B. Kfz-Bordnetz)
\item Ringnetz
\item Maschennetz
\end{itemize}

Spannungshöhe:
\begin{itemize}
    \item Höchstspannung $380/220 \, kV$
    \item Hochspannung $100 \, kV$
    \item Mittelspannung $10/20 \, kV$
    \item Niederspannung $<1 \, kV$
\end{itemize}

\subsection{Trafos}
\ul{Kühlungsart}\\
O: Ölisolierung \qquad N: Natürlich durch Auftrieb\\
A: Luft \qquad W: Wasser \qquad F: Fremd durch Lüfterpumpen

Beispiel: ONAN: Ölkreislauf natürlich, Kühler (Radiatoren) mit natürlicher Kühlleistung\\
ONAF: Ölkreislauf gepumpt, Wasserkühler mit Pumpen\\

\ul{Schaltgruppen}\\
OS: Y, YN, D \qquad US: y, yn, d, zn, a\\
N/n: mit Sternpunktleiter\\
zn: Zick-Zack (nur Sternpunktschaltung)\\
a: Sparwicklung

Beispiel: Dy5 $\rightarrow$ Dreieck OS, Stern US, $5\cdot 30 \degree = 150\degree$ Phasenversatz zwischen OS/US.

\subsection{Freileitungen}
\subsection{Kabel}

\ul{Leiterprofile (Aufbau-Kurzzeichen)}\\
R: Rundleiter \qquad O: Ovalleiter \qquad S: Segmentleiter\\
H: Hohlleiter \qquad E: Eindrähtig \qquad M: Mehrdrähtig \\
V: Verdichtet \quad $\rightarrow$ weitere Kurzzeichen: siehe F34\\

% Betriebskonstanten:

\begin{tabular}[h]{l|l|l|}
    & Freileitung & Kabel\\
    \hline
Reaktanzbelag & groß & klein\\
    \hline
Kapazitätsbelag & klein & groß \\
\hline
Wellenwiderstand & groß & klein \\
\hline
Natürliche Leistung & klein & groß
\end{tabular}

\subsection{Generator}

Polradspannung $U_p$: Im Stator induzierte Spannung\\
Polradwinkel $\vartheta$: Winkel Polrad- zu Klemmenspannung

\begin{tabular}[h]{|l|l|l|}
    \hline
    Aufbau & Stator & Rotor\\
    \hline
    Innenpol & Induktionsspule & Permanentmagnet\\
    \hline
    Außenpol & Permanentmagnet & Induktionsspule\\
    \hline
\end{tabular}

$\Rightarrow$ Drehstrom-Synchronmaschinen als Innenpolgenerator!\\

\subsubsection{Bauart des Rotors}
\begin{itemize}
    \item Schenkelpolrotor $\rightarrow$ Wasserkraftwerke\\
        $100 \min^{-1}\leq n \leq 750 \min^{-1}$
    \item Vollpolrotor (Turbogenerator) $\rightarrow$ Wärmekraftwerke\\
        $n \leq 3000 \min^{-1}$
\end{itemize}

\subsection{Schaltgeräte}
Löschen des Lichtbogens (LiBo):
\begin{itemize}
\item Verlängerung des LiBo
\item Kühlung des LiBo
\item Aufteilen des LiBo
\end{itemize}
