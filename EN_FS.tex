\documentclass[a4paper,11pt]{article}
\usepackage[ngerman]{babel}
\usepackage[utf8]{inputenc}
% \usepackage{german}
\usepackage{amsmath}
\usepackage{amssymb}
\usepackage[left=1.7cm, right=1cm, top=2cm, bottom=1.7cm, showframe]{geometry}

%titel

\title{EN Formelsammlung}
\author{Tony Pham}

\begin{document}
\maketitle
\tableofcontents
\twocolumn

\section{Deckung des Energiebedarfs}
	\subsection{Frequenz-Wirkleistungsregelung}

	\begin{enumerate}
		\item[a)]{stationärer Zustand (Gleichgwicht)}
		\begin{align*}
			 W_{rot} = \frac{1}{2} \cdot J \cdot \omega_{mech} ^{2} &\qquad
			 \omega_{el} = p  \cdot \omega_{mech} \\
			%  \vspace{1pt}\\
			 W_{mech-zu} = W_{el-ab} &\qquad
			 P_{mech-zu} = P_{el-ab}
		\end{align*}

		$p$: Polpaarzahl \quad
		$J$: Massenträgheitsmoment
		
		\item[b)]{Störung}
		
		Sprunghafte Zunahme um $ \Delta P \Rightarrow$ 
		Abbremsen der Rotoren $\Rightarrow$ Sinken der Drehzahl aller Generatoren
		$\Rightarrow$ fehlende Energie wird aus gesp. Rotationsenergie aller elek.
		Maschinen übernommen $\Delta W_{rot}$
		\begin{align*}
			W_{m-zu} \neq W_{el-ab} &\qquad 
			P_{m-zu} \neq P_{el-ab} \\
			\Delta W_{rot} = \frac{1}{2} \cdot & J \cdot (\omega_{stat} - \omega_{akt})
		\end{align*}
	\end{enumerate}

	\subsection{Belastungsdiagramm, -dauer $T_{n}$}

	$W_{el} = P_{n} \cdot T_{a} = P_{max} \cdot T_{m} = P_{mittel} \cdot T_{n}$

\section{Transformatoren}
	\subsection{Grundlagen}
	\subsubsection{Grundgleichungen idealer Trafo}
	% $u_{ind}(t) = -N \cdot \frac{d\phi}{dt} = -N \cdot A \cdot \frac{dB(t)}{dt} = N \cdot A \cdot \omega \cdot \hat{B} \cdot sin (\omega t)$
	
	$U_{ieff} = \frac{1}{\sqrt{2}} \cdot 2 \pi f \cdot N \cdot A_{Fe} \cdot \hat{B} = 4,44 \cdot N \cdot f \cdot \hat{B} \cdot A_{Fe}$
	
	Spannungstrafo:
	$\frac{U_{2}}{U_{1}} = \frac{N_{2}}{N_{1}} = \frac{1}{\text{ü}}$ 

	$\underline{U}'_{2} = \text{ü} \cdot \underline{U}_{2} $

	Stromtrafo:
	$\frac{I_{2}}{I_{1}} = \frac{N_{1}}{N_{2}} = \text{ü}$

	$\underline{I}'_{2} = \underline{I}_{2} \cdot \frac{1}{\text{ü}}$ 
	
	Impedanztrafo:
	$Z_{1} = \text{ü²} \cdot Z_{2}$

	$R'_{2} = \text{ü²} \cdot R_{2}$

	$L'_{2\sigma} = \text{ü²} \cdot L_{2\sigma}$


	% $\underline{U}_{w} = $

	Durchgangsleistung:	
	$S_{1} = S_{2} = S_{D}$

	\subsubsection{Bemessung Trafos}
	
	$N_{1} \cdot A_{L1} = N_{2} \cdot A_{L2}$

	$A_{Lges,1} = A_{Lges,2} = A_{cu,ges} $
	
	Windungsspannung:

	$\dfrac{U_{1}}{N_{1}} = \dfrac{U_{2}}{N_{2}} = U_{W1} = U_{W2} = U_{W}$

	$U_{W} \approx k \cdot A_{Fe}$
	
	Bemessungsleistung 3-Phasen-Trafo:

	$S_{rT} = 3 \cdot 4,44 \cdot f \cdot \hat{B}_{zul} \cdot J_{r} \cdot (A_{Fe} \cdot A_{Cu})$

	$S_{rT} = k \cdot (A_{Fe} \cdot A_{Cu})$

	Auslegung: 
	
	$f = 50Hz, B = 1,7 T - 1,8 T, A_{Fe} = 1,2 \frac{A}{mm^{2}}$

	\subsubsection{Wachstumsgesetze}

	Frage: 1 Trafo (900MVA) wirtschaftlicher als 3 Trafos (je 300 MVA)?

	$A_{Fe,neu} = A_{Fe} \cdot k^2$ \quad $A_{cu,neu} = A_{cu} \cdot k^2$ \quad

	$V_{neu} = V \cdot k^3$ \quad $S_{rT,neu} = S_{rT} \cdot k^4$

	Gewicht: $m_{neu} = m \cdot k^3 \Rightarrow$ Kosten $\downarrow$ 

	Verluste: $P_{V,Cu,neu} = P_{V,Cu} \cdot k^3$

	Verluste: $P_{V,Fe,neu} = P_{V,Fe} \cdot k^3$

	$P_{V,ges,neu} = P_{V,ges} \cdot k^3 \Rightarrow$ Wirkungsgrad $\eta \uparrow$

	Kühlung: $A_{Huell,neu} = A_H \cdot k^2$ 

	$\Rightarrow$ relative Kühlfläche $\downarrow$, Aufwand Kühlung $\uparrow$

	1 Trafo wirtschaftlicher!

	\subsection{ESB Trafo}

	Maschengleichung:
	$
	\begin{pmatrix}
		R_{1} & j\omega L'_{1\sigma} \\
		R'_{2} & j\omega L'_{2\sigma}
	\end{pmatrix} \cdot
	\begin{pmatrix}
		\underline{I}_{1} \\
		-\underline{I}'_{2}
	\end{pmatrix} =
	\begin{pmatrix}
		\underline{U}_{1} \\
		\underline{U}'_{2}
	\end{pmatrix}
	$

	$U_{1} = R_{1} \cdot \underline{I}_{1} + j\omega L_{1\sigma} \cdot \underline{I}_{1} - R'_{2}- \underline{I}'_{2}-j\omega L'_{2\sigma} \cdot \underline{I}'_{2}+ \underline{U}'_{2}$

	Mit $I_{1} = -I_{2}$ und $U'_{2} = \text{ü} \cdot U_2$:

	$U_{1} = (R_1 + R_2) \cdot \underline{I}_1 + j(X_{1\sigma} + X_{2\sigma}) \cdot \underline{I}_1 + \text{ü} \cdot \underline{U}_2$

	Mit $R_T = R_1 + R_2$ und $X_T = X_{1\sigma} + X_{2\sigma}$:
	
	$U_1 = (R_T + jX_T) \cdot I_1 + \text{ü} \cdot U_2$

	\subsection{Betriebskonstanten}
	\subsubsection{Kurzschlussmessung}

	Kurzschluss(KS) der Klemmen an  \textbf{OS}/US
	
	$\Rightarrow$ Messen der Spannung an \textbf{US}/OS 
	
	$\Rightarrow$ Angabe relative KS-Spannung $u_{k}$ in $\%$

	komplexe KS-Spannung: $\underline{U}_{kn} = U_{kn,R} +jU_{kn,X}$

	$u_{k1} = \frac{U_{k1}}{U_{rT1}/ \sqrt{3}}$

	$u_{k1} = u_{k2} = u_{k} = u_{kR}+j u_{kX}$

	KS-Strom:
	$I_k = \frac{I_{rT}}{u_k}$

	KS-Spannung:
	$U_k = u_k \cdot U_{rT}$
	\subsubsection{Berechnung Betriebskonstanten}
	
	Betrag Impedanz $Z$:
	
	$Z_T = u_k \cdot \frac{U_{rT}^2}{\sqrt{3}I_{rT}U_{rT}} =\frac{u_k \cdot U_{rT}^2}{100 \% \cdot S_{rT}} = \frac{u_k}{100\%} \cdot Z_{Bezug}$

	Resistanz $R$:
	
	$R_T = u_{kR} \cdot \frac{U_{rT}}{\sqrt{3} I_{rT}} = \frac{u_{kR}\cdot U^2_{rT}}{100\% \cdot S_{rT}} = P_{kT} \cdot \frac{U_{rT}^2}{S_{rT}^2}$

	$u_{kR} = \frac{P_{kT}}{S_{rT}}$

	$u_{kX} = \sqrt{u^2_k - u^2_{kR}}$
	
	Reaktanz $X$:
	
	$X_T = \sqrt{Z^2_T - R^2_T} = u_{kX} \cdot \frac{U^2_{rT}}{S_{rT}}$
	
	Leistung Kurschlussverlust $P_{kT}$:
	
	$P_{kT} \approx 3\cdot R_T \cdot I^2_{rT}$
	
	\subsection{Trafo-Schaltgruppen}
	tableofcontents

	
	tableofcontents
	
	
	tableofcontents

	\subsection{Parallelbetrieb Trafos}
	Allgemein:

	$\underline{U}_{1,T1} = \underline{U}_{1,T2} = \underline{U}_{1T}$

	$\underline{U}_{2,T1} = \underline{U}_{1T} \cdot $

	% $$
	% $$
	% \begin{align*}
	
	% \end{align*}



	

\end{document}

