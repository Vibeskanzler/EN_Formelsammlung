% \clearpage
\section{Betrieb von Leitungen}
\subsection{Kenngrößen}

\begin{itemize}
    \item[] \textbf{Leitung mit Verlusten}
    \begin{gather*}
        \ul{\gamma} = \sqrt{(R'_b +jX'_b)\cdot(G'_b+jB'_b)}= \alpha \cdot j \beta \left[\frac{\degree}{km} \right]\\
        \ul{Z}_w = \sqrt{\dfrac{R'_b + jX'_b}{G'_b + jB'_b}}= |Z_w| \cdot e^{j\delta}
    \end{gather*}
    \indent Falls Formel von $\ul{Z}_w$ nicht über TR berechenbar $\rightarrow$\\
    Betrag: erst $\ul{Z}_w^2$, dann $\sqrt{|Z_w^2|}$ ermitteln \\
    Phase: $0.5 \cdot \arg(\ul{Z}_w^2)$ \\

    \indent $\gamma:$ Ausbreitungskonstante\\
    \indent $\alpha:$ Dämpfungskonstante\\
    \indent $\beta:$ Phasenkonstante $[\frac{rad}{km} = \frac{180\degree}{\pi} \cdot \frac{1}{km} = \frac{\degree}{km}]$\\
    \indent $Z_w:$ Wellenwiderstand\\
    \indent $\delta:$ Phase des Wellenwiderstandes\\

    \item[] \textbf{Leitung ohne Verluste}\\
            $R'_b = G'_b = 0\rightarrow \alpha, \delta = 0$
        \begin{gather*}
            \ul{\gamma} = j \beta = j \sqrt{X'_b \cdot B'_b} = j \omega \cdot \sqrt{L'_b \cdot C'_b}\left[\frac{\degree}{km} \right]\\
            |\beta| = \sqrt{X'_b \cdot B'_b} \cdot \frac{180\degree}{\pi} = \omega \cdot \sqrt{L'_b \cdot C'_b} \\
            |Z_w| = \sqrt{ \dfrac{X'_b}{B'_b} } = \sqrt{ \dfrac{L'_b}{B'_b} }
        \end{gather*}
    Richtwerte:
            $Z_w \approx 400\: \ohm \qquad \beta = \frac{0,06\degree}{km}$\\

    \item[] \textbf{natürliche Leistung, Blindleistungsverluste}

        - gilt bei Leitung ohne Verlusten, DS-System\\
        - natürlicher Betrieb bei $Q_L = Q_C$
        \begin{gather*}
            I_{nat} = \frac{U_{LE}}{\sqrt{X_L/B_L}} =  \frac{U_{LE}}{Z_w} \neq f(l)\\
            P_{nat} = 3 \cdot U_{LE} \cdot I_{nat} = \frac{U^2_{LL}}{Z_w} = \frac{3 \cdot U_{LE}}{Z_w}\\
            Q_L = 3 \cdot X_L \cdot I_L^2 \qquad Q_C = 3 \cdot B_L \cdot U^2_{LE}\\
            \frac{Q_V}{Q_C} = \left(\frac{S_u}{P_{nat}} \right)^2-1\\
            Q_v = Q_1 - Q_2 = Q_L - Q_C = Q_C \cdot (\frac{Q_L}{Q_C}-1)\\
            S_v = S_1 - S_2 = P_v + j Q_v\\
            S_u = 3 \cdot U_{LE} \cdot I_L = \sqrt{3} \cdot U_{LL} \cdot I_L
        \end{gather*}

        \indent $Q_v:$ Blindleistungsverluste\\
        \indent $S_u:$ Übertragungsscheinleistung

\subsection{Ersatzschaltbilder (ESB)}

\textbf{Kenngrößen}\\

\begin{tabular}[h]{c|c|c}
    & Längszweig & Querzweig $\parallel$ \\
    \hline
    Wirk & $R_L = R'_b \cdot l$ & $G_L= G'_b \cdot l$\\
    Blind & $X_L = X'_b \cdot l$ & $B_L = B'_b \cdot l$\\
    % Blind & $X_L = X'_b \cdot l$ & $B_L = B_'b \cdot l$\\
\end{tabular}\\

$R_L$: Resistanz \qquad $X_L$: Reaktanz\\
$B_L$: Suszeptanz \qquad $G_L$: Konduktanz\\

Index $1/\ul{2}$: Größe am Anfang/\ul{Ende} der Leitung\\
Index $L$: Größen bezogen auf Leitung\\
$dU$: Spannung am Längszweig\\

\newpage
\item[] \textbf{MS-/NS-Leitungen mit Verlusten}\\
    $I_G, I_C<< I_L \Rightarrow G'_b = B'_b = 0$
        \begin{gather*}
            \ul{I_1} = \ul{I}_L = \ul{I}_2 \qquad
            \ul{U}_1 = d \ul{U} + \ul{U}_2\\
            d \ul{U} = (R_L + jX_L) \cdot \ul{I}_L = (R'_b +jX'_b) \cdot l \cdot \ul{I}_L\\
            \ul{Z}_L = R_L +jX_L \qquad \varphi_Z =\arctan \frac{X_L}{R_L} = \arctan \frac{X'_b}{R'_b}
        \end{gather*}

\item[] \textbf{\ul{Kurze} HS-/HöS-DS-Freileitungen}

    $U_{LL}$ > 100 kV für $l\leq$ 220 km\\

    \ul{ohne Verluste:} ($R'_b = G'_b = 0$)
    \begin{gather*}
        B'_1 = B'_2 = \frac{B'_b}{2} \qquad
        \ul{I}_{C1/2} = j \: B_{1/2} \cdot \ul{U}_{1/2}\\
        \ul{I}_1 = \ul{I}_{C1} + \ul{I}_L \qquad  \ul{I}_L =  \ul{I}_{C2} +  \ul{I}_2
         % \ul{U}_1 = d\ul{U} + \ul{U}_2 = (R_L + jX_L) \cdot \ul{I}_L + \ul{U}_2\\
    \end{gather*}
    \textit{bei Leerlauf:} $\ul{I}_2 = 0 \rightarrow \ul{I}_{C2} = \ul{I}_L$
    \begin{align*}
        \ul{U}_1 &= (1 - \frac{X_L\cdot B_L}{2}) \cdot \ul{U}_2\\
        \ul{I}_1 &=  \ul{I}_2 + j \: 0.5 \cdot B_L \cdot ( \ul{U}_1 + \ul{U}_2)\\
        Q_1 &= 3\cdot U_1 \cdot I_1
    \end{align*}
    $\ul{I}_1$: Ladestrom \qquad
    $Q_1:$ Ladeleistung\\

    \textit{bei Betrieb mit natürlicher Leistung}: $R_2 = Z_w = \ul{Z}_2$
        \begin{gather*}
            S_1 = S_2 = P_1 = P_2 = P_{nat} \quad
            |\ul{U}_1| = |\ul{U}_2| \quad |I_1| = |I_2|\\
            \ul{U}_1 = \ul{U}_2 \cdot(1-\frac{1}{2} \cdot B_L \cdot X_L + j\: \frac{X_L}{Z_w})\\
            \ul{I}_1=  \frac{\ul{U}_2}{Z_w}+j\: \frac{B_L}{2}\cdot (\ul{U}_1 + \ul{U}_2)\\
            \ul{I}_L=  \frac{\ul{U}_2}{Z_w}+j\: \frac{B_L}{2}\cdot \ul{U}_2\\
        \end{gather*}

\item[] \textbf{\ul{Lange} HS-/HöS-DS-Freileitungen}

    $U_{LL}$ > 100 kV für $l>$ 220 km\\

    \ul{ohne Verluste} ($R'_b = G'_b = 0$)
    \begin{align*}
        \ul{U}_1 &= \ul{U}_2 \cdot \cos(\beta l) + j \cdot \ul{I}_2 \cdot Z_w \cdot \sin (\beta l)\\
        \ul{I}_1 &= \ul{I}_2 \cdot \cos(\beta l) + j \cdot \frac{\ul{U}_2}{Z_w} \cdot \sin (\beta l)
    \end{align*}
    \textit{bei Leerlauf:} $\ul{I}_2 = 0$, $\ul{Z}_2 \rightarrow \infty$\\
    \textit{bei Betrieb mit natürlicher Leistung:} $\varphi = \beta l$
    \begin{gather*}
        \ul{U}_1 = |\ul{U}_2| \cdot e^{j\varphi} \qquad
        \ul{I}_1 = |\ul{I}_2| \cdot e^{j\varphi}\\
        \ul{S}_1 = \ul{S}_2 = \frac{3 \cdot U_{2,LE}^2}{Z_w} = P_{nat}
    \end{gather*}

    \ul{mit Verlusten}
    \begin{align*}
        \ul{U}_1 &= \ul{U}_2 \cdot \cosh(\ul{\gamma} l) + \ul{I}_2 \cdot \ul{Z}_w \cdot \sinh(\ul{\gamma} l)\\
        \ul{I}_1 &= \ul{I}_2 \cdot \cosh(\ul{\gamma} l) + \frac{\ul{U}_2}{\ul{Z}_w}\cdot \sinh(\ul{\gamma} l)
    \end{align*}
    nicht direkt mit komplexen Modus des TR einsetzbar!\\
    Lösung: $\alpha \cdot l $ und $ \beta \cdot l [\degree]$ einzeln berechnen, dann:
    \begin{gather*}
        \cosh(\ul{\gamma}l) = \frac{1}{2} \left[e^{\alpha l} \cdot e^{j\beta l} + e^{-\alpha l} \cdot e^{-j\beta l} \right]\\
        \sinh (\ul{\gamma}l)= \frac{1}{2} \left[e^{\alpha l} \cdot e^{j\beta l} - e^{-\alpha l} \cdot e^{-j\beta l} \right]\\
        % \ul{U}_1 = \ul{U}_2 \cdot  \frac{1}{2} \left[e^{\alpha l} \cdot e^{j\beta l} + e^{-\alpha l} \cdot e^{-j\beta l} \right]\\
        % + \ul{I}_2 \cdot \ul{Z}_w \cdot \frac{1}{2} \left[e^{\alpha l} \cdot e^{j\beta l} - e^{-\alpha l} \cdot e^{-j\beta l} \right]\\
        % \ul{I}_1 = \ul{I}_2 \cdot  \frac{1}{2} \left[e^{\alpha l} \cdot e^{j\beta l} + e^{-\alpha l} \cdot e^{-j\beta l} \right]\\
        % + \ul{I}_2 \cdot \ul{Z}_w \cdot \frac{1}{2} \left[e^{\alpha l} \cdot e^{j\beta l} - e^{-\alpha l} \cdot e^{-j\beta l} \right]\\
    \end{gather*}

\item[] \textbf{HS-/HöS-DS-Kabel}

    $l>95km$: langes Kabel\\
    $Q_V = Q_L - Q_C$ wie Freileitungen (FL), aber $Q_K > Q_{FL}$
    \begin{gather*}
        S_{th}= 3\cdot U_{LE}\cdot I_{Dauer}\\
        P_{max}= \sqrt{S^2_{th}-Q^2_V} = \sqrt{S^2_{th} - Q'^2_V \cdot l^2}\\
        Q_V = Q'_V \cdot l \qquad l_{max}= \frac{S_{th}}{Q'_v}
    \end{gather*}

    $S_{th}$: thermisch, max. Scheinleistung\\
    $P_{max}$: max. übertragbare Wirkleistung\\
    $l_{max}$: max. Kabellänge, wenn $P_{max} = 0$

\end{itemize}
