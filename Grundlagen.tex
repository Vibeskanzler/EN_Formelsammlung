\section{Deckung des Energiebedarfs}
	\subsection{Frequenz-Wirkleistungsregelung}

	\begin{enumerate}
		\item[a)]{stationärer Zustand (Gleichgwicht)}
		\begin{align*}
			 W_{rot} = \frac{1}{2} \cdot J \cdot \omega_{mech} ^{2} &\qquad
			 \omega_{el} = p  \cdot \omega_{mech} \\
			%  \vspace{1pt}\\
			 W_{mech-zu} = W_{el-ab} &\qquad
			 P_{mech-zu} = P_{el-ab}
		\end{align*}

		$p$: Polpaarzahl \quad
		$J$: Massenträgheitsmoment

		\item[b)]{Störung}

		Sprunghafte Zunahme um $ \Delta P \Rightarrow$
		Abbremsen der Rotoren $\Rightarrow$ Sinken der Drehzahl aller Generatoren
		$\Rightarrow$ fehlende Energie wird aus gesp. Rotationsenergie aller elek.
		Maschinen übernommen $\Delta W_{rot}$
		\begin{align*}
			W_{m-zu} \neq W_{el-ab} &\qquad
			P_{m-zu} \neq P_{el-ab} \\
			\Delta W_{rot} = \frac{1}{2} \cdot & J \cdot (\omega_{stat} - \omega_{akt})
		\end{align*}
	\end{enumerate}

	\subsection{Belastungsdiagramm, -dauer $T_{n}$}

\[ W_{el} = P_{n} \cdot T_{a} = P_{max} \cdot T_{m} = P_{mittel} \cdot T_{n} \]