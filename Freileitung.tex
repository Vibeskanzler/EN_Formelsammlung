\section{Freileitung}
\subsection{Durchhang von Freileitungsseilen}
hängen hyperbolisch durch. (ab \SI{110}{k\volt}:)
\begin{equation*}
    h_{min} = \SI{6}{\metre} + \left(\frac{U_{\mathit{nLL}} - \SI{110}{k\volt}}{\SI{150}{k\volt}}\right)m
\end{equation*}
\subsection{Resistanzbelag}
\begin{itemize}
\item[]{\textbf{DC-Widerstand}}

$A_{\mathit{eff}}:$ Wirksamer Querschnitt [$mm^2$]\\
$F_{\vartheta}:$ Widerstandserhöhung durch Erwärmung\\
$\vartheta_{max}:$ max. zul. Betriebstemp. des Leiterseils
\begin{gather*}
    R_{\textcolor{purple}{=}}' = \frac{R_{=}}{l} = \frac{\rho_{20 \degree}}{A_{\mathit{eff}}}\cdot\frac{1}{km}\\
    F_{\vartheta} = 1 + \alpha (\vartheta_{max} - 20\degree C)
    % + \beta (\vartheta_{max} - 20\degree C)...
\end{gather*}

\begin{table}[h]
\centering
\begin{tabular}[h]{|l|c|c|}
    \hline
    Material & $\rho_{\SI{20}{\degreeCelsius}}$ in $\frac{m\ohm \cdot mm^2}{m}$ & $\alpha$ in $K^{-1}$ \\
    \hline
    Alu & 28,6 & 0,0038 \\
    \hline
    Kupfer & 17,8 & 0,0039 \\
    \hline
    Silber & 16 & 0,0038 \\
    \hline
    Eisendraht & 120 & 0,0052 \\
    \hline
\end{tabular}
\end{table}
weitere Kenngrößen siehe F39\\

\item[]{\textbf{AC-Widerstand}}
% \textbf{Stromdichte (Skineffekt)}\\
% \textbf{Eindringtiefe}\\
% \textbf{Gesamt Betriebswiderstand}
\begin{gather*}
    J = J \cdot e^{-x/\delta}\\
    \delta = \sqrt{ \frac{\rho}{\pi \cdot \mu_{0} \cdot f}}
     = \sqrt{ \frac{1}{\pi \cdot \kappa \cdot \mu_{0} \cdot f}}
\end{gather*}
$J:$ Stromdichte (Leiterrand)\\
$x:$ Abstand vom Leiterrand (Oberfläche)\\
$\delta:$ Eindringtiefe (Skineffekt)\\

\item[]{\textbf{Betriebs-Resistanzbelag}}
\begin{gather*}
    R_{b}' = R_{=}' \cdot F_{\vartheta} \cdot F_{S} = \frac{R_{b\mathtt{Seil}}'}{n_{\mathtt{Seil}}}
\end{gather*}
$F_S:$ Widerstandserhöhung durch Skineffekt
\newpage

\subsection{Induktivität}
\textbf{Aüßere Ind. Einzelleiter} \textcolor{dgreen}{mag. Fluss}\\
\textbf{-||- Doppelleiter}\\
\textbf{Innere Ind. Einzelleiter} \textcolor{dgreen}{verketteter mag. Fluss}\\
\textbf{-||- Doppelleiter}
\begin{gather*}
    \Phi_{a1} = \frac{\mu I l}{2 \pi} \cdot \ln \left( \frac{D-r}{r}\right)\\
    L_{a} = \frac{2\Phi_{a1}}{I} = \frac{\mu l}{2 \pi} \cdot \ln \left( \frac{D}{r}\right)\\
    \Psi_{i1} = \Psi_{i2} = \frac{\mu I l}{8 \pi}\\
    L_{i} = \frac{2\Psi_{i1}}{I} = \frac{\mu l}{8 \pi}
\end{gather*}

\textbf{gesamt Induktivität}\\
\textcolor{dgreen}{$\mu = \mu_{0}, D \gg r$}
\begin{gather*}
    L_{ges} = L_{a} + L_{i} = \frac{\mu l}{2\pi} \left( \ln \left(\frac{D}{r}\right) + \frac{1}{4}\right)\\
    L' = \frac{L_{ges}}{l} = \frac{\mu }{2\pi} \left( \ln \left(\frac{D}{r}\right) + \frac{1}{4}\right)
\end{gather*}

\subsection{Reaktanzbelag}
Gleichungen für $f=50 \mathit{Hz}$\\
Metallmantel keine Schirmung! Für $D$ nicht $\gg r$!\\
Radius r-Werte siehe F39\\

\textbf{2-Phasig/Wechelstrom}
\begin{gather*}
    L'_b = 4\cdot 10^{-7} \left[ \ln \left( \frac{D_m}{r} \right) + 0,25 \right] \left[\frac{H}{m} \right] \\
    X_{b}'= \omega L'_b = \pi \left[ 4 \ln \left( \frac{D_m}{r}\right) +1 \right] \cdot 10 ^{-2}   \left[\frac{\Omega}{km}\right]
\end{gather*}

\textbf{Drehstrom}

    \item[]{\ul{Einfach-/Bündelleiter}}
\begin{align*}
    X_{b}' &= \frac{\pi}{2} \left( 4 \ln \left( \frac{D_{m}}{r_B}\right) +\frac{1}{n} \right) \cdot 10 ^{-2}   \left[\frac{\Omega}{km}\right]\\
    \vspace{5pt}
    D_{m} &= \sqrt[3]{D_{12} \cdot D_{23} \cdot D_{31}}\\
    r_B &= \sqrt[n]{n \cdot r \cdot r_{T}^{n-1}}\\
    r_T &= \frac{a}{2 \sin \frac{180 \degree}{n}}
\end{align*}

$n:$ Anzahl Teilleiter (wenn $n>1 \Rightarrow$ Bündelleiter)\\
$D_m$: Abstände bei Symmetrie der Phasen zur Mastmitte\\
$D_{12}:$ Abstand $L1 - L2$ usw.\\
$r_B:$ Ersatzradius (wenn $n=1$, dann $r_B = r$)\\
$r_T:$ Radius Teilleiter (bei $n>1$)\\
$a:$ Abstand Teilleiter (bei $n>1$)\\

% Werden Bezeichnet nach: n x 2r / a


\item[]{\ul{Doppelleiter}}
\begin{gather*}
    X_{b}' = \frac{\pi}{2} \left( 4 \ln \left( \frac{D_{m} \cdot D_{L1/L\RN{2}}}{r \cdot D_{L1/L\RN{1}}}\right) + 1 \right) \cdot 10 ^{-2}   \left[\frac{\Omega}{km}\right]\\
    D_{L1/L\RN{2}} = \sqrt[3]{D_{1\RN{2}} \cdot D_{2\RN{3}} \cdot D_{3\RN{1}}}\\
    D_{L1/L\RN{1}} = \sqrt[3]{D_{1\RN{1}} \cdot D_{2\RN{2}} \cdot D_{3\RN{3}}}
\end{gather*}
Bei Asymmetrie (Phase zur Mastmitte)
\begin{equation*}
    D_{L1/L\RN{2}} = \sqrt[6]{D_{1\RN{2}} \cdot D_{2\RN{3}} \cdot D_{3\RN{1}} \cdot D_{1\RN{3}} \cdot D_{2\RN{1}} \cdot D_{3\RN{2}}}
\end{equation*}
\end{itemize}

\subsection{Suszeptanzbelag (Blindleitwert)}
\begin{itemize}
    \item[]{\textbf{2-Phasig/Wechselstrom}}
    \begin{align*}
        B'_b = \omega \cdot C'_b = \frac{17,47}{ln\left(\dfrac{D_m}{r_B} \right)} \left[ \frac{\mu S}{km} \right]
    \end{align*}
\end{itemize}
\clearpage
