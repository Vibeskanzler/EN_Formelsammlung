\section{Grundlagen}
\subsection{Drehstrom (DS), 3-Phasen-System}
\begin{enumerate}


    \item{\textbf{Spannungen in DS (symmetrisch)}}\\
        \underline{Leiter-Erde-Spannung $U_{LE}=230V$}
        \begin{flalign*}
            \underline{U}_{L1} &= U_{LE} \angle 0^\circ &\\
            \underline{U}_{L2} &= U_{LE} \angle -120^\circ = U_{LE} \angle 240^\circ &\\
            \underline{U}_{L3} &= U_{LE} \angle -240^\circ = U_{LE} \angle 120^\circ &
        \end{flalign*}

        \underline{Leiter-Leiter-Spannung $U_{LL}=400V$}
        \begin{flalign*}
            U_{LL} &= U_{LE} \cdot \sqrt{3} &\\
            \underline{U}_{12} &= \underline{U}_{L1} - \underline{U}_{L2} = U_{LL} \angle 30^\circ &\\
            \underline{U}_{23} &= \underline{U}_{L2} - \underline{U}_{L3} = U_{LL} \angle 270^\circ &\\
            \underline{U}_{31} &= \underline{U}_{L3} - \underline{U}_{L1} = U_{LL} \angle 150^\circ &
        \end{flalign*}

    \item{\textbf{Ströme in DS (symmetrisch)}}
        \begin{flalign*}
            \underline{I}_{Lx} &= \dfrac{\underline{U}_{Lx}}{\underline{Z}} &
       \end{flalign*}

        $Lx:$ Stranggrößen L1, L2, L3\\

    \item{\textbf{Effektivgrößen, Symmetrische Last}}
        \begin{center}
            \vspace{-1em}
    \begin{tabular}[h]{l|l|l}
        Stranggröße & Stern & Dreieck
     \vspace{1pt}\\
        Spannung $U_{LE}$ & $U_{LE} = \dfrac{U_{LL}}{\sqrt{3}}$ & $U_{LE}=U_{LL}$\\
        Strom $I_{str}$ & $I_{str} = I_r$ & $I_{str} = \dfrac{I_r}{\sqrt{3}}$
    \end{tabular}
    \vspace{0.5em}\\
    $I_{r}:$ Zuleitungs-, Betriebs-, Bemessungssstrom
    \end{center}

    \item{\textbf{Leistungen in DS}}

        \underline{Scheinleistung S [VA]:}
        \begin{flalign*}
        S &= 3 \cdot U_{LE} \cdot I_L
          = \sqrt{3} \cdot U_{LL} \cdot I_L &\\
          &= \sqrt{P^2 + Q^2} &\\
        \underline{S} &= 3 \cdot \underline{U}_{LE} \cdot \underline{I}_L^*
                      = P+jQ &
        \end{flalign*}
        \textit{in Sternschaltung:}
        \begin{flalign*}
        \underline{S}_{ds} &= \frac{U_{LL}^2}{\underline{Z}_{LN}*}  &\\
        \underline{S}_{ws}  &= \frac{U_{LL}^2}{3 \cdot \underline{Z}_{LN}*} &
        \end{flalign*}

        \underline{Wirkleistung P [W]:}
        \begin{flalign*}
        P   &= S \cdot \cos \varphi &\\
            &= 3 \cdot U_{LE} \cdot I_L \cdot \cos \varphi &\\
            &= \sqrt{3} \cdot U_{LL} \cdot I_L \cdot \cos \varphi &
        \end{flalign*}

        \underline{Blindleistung Q [var]:}
        \begin{flalign*}
        |Q| &= S \cdot \sin \varphi = P \cdot \tan \varphi &\\
            &= 3 \cdot U_{LE} \cdot I_L \cdot \sin \varphi &\\
            &= \sqrt{3} \cdot U_{LL} \cdot I_L \cdot \sin \varphi &\\
           & Q
            \begin{cases}
                \text{induktiv} >0\\
                \text{kapazitiv} <0
                \end{cases}
        \end{flalign*}
\end{enumerate}

\subsection{Energiebedarf, Deckung}
\begin{enumerate}
    \item{\textbf{Tagesbelastungskurve}}
\begin{flalign*}
    W &= \int_{0}^{T_{n}} P(t)  \,dt \\
      &= P_{n} \cdot T_{a} = P_{max} \cdot T_{m} =
    P_{mittel} \cdot T_{n}
\end{flalign*}

\begin{tabular}[h]{l|l}
    \hline
    $P_{n}$         & Nennleistung \\
    $T_{a}$         & Ausnutzungsdauer \\
    \hline
    $P_{max}$       & Höchstlast \\
    $T_{m}$         & Benutzungsdauer \\
    \hline
    $P_{mittel}$    & mittlere Leistung \\
    $T_{n}$         & Nennbetriebsdauer (meist. 24h) \\
\end{tabular}\\
\\
\item{\textbf{Frequenz-Wirkleistungs-Regelung}}
    		\item[a)]{\underline{stationärer Zustand (Gleichgewicht)}}
		\begin{align*}
			 W_{rot} = \frac{1}{2} \cdot J \cdot \omega_{mech} ^{2} &\qquad
			 \omega_{el} = p  \cdot \omega_{mech} \\
			%  \vspace{1pt}\\
			 W_{mech-zu} = W_{el-ab} &\qquad
			 P_{mech-zu} = P_{el-ab}
		\end{align*}

		$p$: Polpaarzahl \quad
		$J$: Massenträgheitsmoment\\

		\item[b)]{\underline{Störung}}
		% Sprunghafte Zunahme um $ \Delta P \Rightarrow$
		% Abbremsen der Rotoren $\Rightarrow$ Sinken der Drehzahl aller Generatoren
		% $\Rightarrow$ fehlende Energie wird aus gesp. Rotationsenergie aller elek.
		% Maschinen übernommen $\Delta W_{rot}$
		\begin{align*}
			W_{m-zu} \neq W_{el-ab} &\qquad
			P_{m-zu} \neq P_{el-ab} \\
			\Delta W_{rot} = \frac{1}{2} \cdot & J \cdot (\omega_{stat} - \omega_{akt})\\
		\end{align*}

\end{enumerate}

% \begin{gather*}
%     [R'] = [X'] = \frac{\Omega}{km} = \frac{m\Omega}{m} \\
%     [G'] = \frac{nS}{km} = \frac{pS}{m} \\
%     [B'] = \frac{\mu S}{km} = \frac{nS}{m}\\
%     B_{b}' = \omega C_{b}'
% \end{gather*}